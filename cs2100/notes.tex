\documentclass{article}
\usepackage[a4paper,margin=0.5in,landscape]{geometry}
\usepackage[]{amsmath}
\usepackage{multicol}
\usepackage{savetrees}
\title{CS2100 Notes}
\author{Conrad Soon}
% \date{\today}
\begin{document}
\begin{multicols*}{3}
\maketitle
\section{Week 1}
\subsection{Base-n Number Systems}
\subsubsection{Integer Decimal to Base-$n$ Conversion}
Repeated division by $n$, remainder forms the $n$-digit of the number in base-$n$.
\subsubsection{Fractional Decimal to Base-n Conversion}
Repeated multiplication by $n$, ones-place forms the $n$-digit of the number in base-$n$.
\subsection{Number Representation Schemes}
\subsubsection{Sign and Magnitude}
MSB is sign bit, $0$ denotes positive, $1$ denotes negative. Remaining bits form the magnitude.
\subsubsection{One's Complement}
MSB is sign bit, $0$ denotes positive, $1$ denotes negative.
Negative numbers by inversion of all bits.
Arithmetic done by adding carry-out of MSB back into sum.
\subsubsection{Two's Complement}
MSB is sign bit, $0$ denotes positive, $1$ denotes negative.
Negative numbers by inversion of all bits, followed by addition of $1$.
Arithmetic done by ignoring carry out of MSB.
\subsubsection{Excess-$n$ Representation}
Number $N$ represents value $N-n$. For example, $000$ in excess-$4$ representation represents $0-4=-4$.

\subsection{Real Number Representations}
\subsubsection{IEEE-754 Float}
MSB is sign bit, $0$ denotes positive, $1$ denotes negative.
Next $8$ bits are exponent bits, represented in excess-127.
Next $23$ bits are magnitude bits, normalised with implicit leading bit of 1.
$x = (-1)^{S}*M*2^{(E-127)}$
\section{Week 2}
\end{multicols*}
\end{document}